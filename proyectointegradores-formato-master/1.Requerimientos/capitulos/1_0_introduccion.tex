\chapter{Introducción}

Al menos 3 de millones de peruanos poseen un nivel de colesterol alto, el cual pocas personas suelen ser diagnosticadas o tratadas a tiempo de que se empeore. Por ello, en el Perú el colesterol es uno de los problemas que más afecta a las personas. Además, en la 35ª Conferencia Regional para América Latina y el Caribe de la Organización de las Naciones Unidas para la Alimentación y la Agricultura (FAO), anuncia que Perú había alcanzado el tercer lugar en sobre peso y obesidad en la región.

\section{Propósito}

Implementar un sitio web y un aplicativo móvil que ayude a controlar y bajar el riesgo de que la persona llegue a un nivel de colesterol alto, de tal forma que al cliente se le proporcionará una variedad de platos que no afecten a su salud y la recomendación de ejercicios que se pueden realizar desde su casa, esto se le proporcionará dependiendo el nivel de colesterol que obtendrá al momento del proporcionarnos sus datos.

\section{Alcance}

\begin{itemize}
	\item La aplicación estará implementada con un formulario diseñado para el ingreso de datos, los datos que serán ingresados serán los proporcionados en un análisis de sangre.
	
	\item	La app se encargará de procesar los datos ingesados y compararlos con una tabla de uso general (la que contiene los niveles de colesterol considerados como inofensivos por escalas según edad y sexo) generando un diagnostico.
	
	\item Se habilitara el acceso a una base de datos que contiene platos de comida, con las características necesarias para la disminución del colesterol, los que podrann ser discriminados segun los gustos del usuario.
	
	\item  Se habilitara el acceso a un conjunto de rutinas de ejercicio físico, los cuales también podrán ser filtrados de acuerdo con la dificultad, tiempo y estilo de vida.
	
	\item La aplicacion creara un regstro con todos los analisis realizados para llevr un control de la salud del usuario.
\section{Definiciones, siglas y abreviaturas}

\section{Referencias}